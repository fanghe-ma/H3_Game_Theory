\documentclass[10pt, a4paper]{article}
\usepackage[margin = 1in]{geometry}
\usepackage{amsmath}
\setlength{\parindent}{0em}
\begin{document}

\title{Auctions info-pack}

\section{Summary}
\begin{enumerate}
   \item For First Price sealed bid, perfect info, 
      \begin{itemize}
         \item Equilibrium bidding strategy in non-dominated strategies
             \begin{itemize}
                \item highest bidder bids second highest v 
                \item second highest bidder bids $v - \delta$
                \item all others bid less than their v, above 0
             \end{itemize}
         \item Expected revenue for seller
             \begin{itemize}
                \item second highest v
             \end{itemize}
      \end{itemize}
   \item For Second Price sealed bid, perfect info, 
      \begin{itemize}
         \item Equilibrium bidding strategy in weakly-dominated strategies
             \begin{itemize}
                \item everyone bids their own v
             \end{itemize}
         \item Expected revenue for seller 
             \begin{itemize}
                \item second highest v
             \end{itemize}
      \end{itemize}
   \item For First Price sealed bid, imperfect info, 
      \begin{itemize}
         \item Equilibrium bidding strategy in non-dominated strategies
             \begin{itemize}
                \item expected second highest v conditional on having the highest v =\[\frac{n-1}{n}v_i\]
             \end{itemize}
         \item Expected revenue for seller 
             \begin{itemize}
                \item expected second highest v = \[ \frac{n-1}{n+1}\]
             \end{itemize}
      \end{itemize}
   \item For Second Price sealed bid, imperfect info, 
      \begin{itemize}
         \item Equilibrium bidding strategy in weakly-dominated strategies
             \begin{itemize}
                \item bids their own v, $v_i$
             \end{itemize}
         \item Expected revenue for seller 
             \begin{itemize}
                \item expected second highest v = \[ \frac{n-1}{n+1}\]
             \end{itemize}
      \end{itemize}
   \item For All Pay Auction, perfect info, common value
      \begin{itemize}
         \item Equilibrium bidding strategy 
             \begin{itemize}
                \item no equilibrium in pure strategies 
                \item equilibrium in mixed strategies when everyone bids 
                   \[
                      b_i \sim U(0, v_{common})
                   \]
             \end{itemize}
          \item Expected revenue for seller = \[v_{commmon}\]
      \end{itemize}
   \item For All Pay Auction, perfect info, private value
      \begin{itemize}
         \item Equilibrium bidding strategy 
             \begin{itemize}
                \item no equilibrium in pure strategies 
                \item equilibrium in mixed strategies when  
                   \begin{align*}
                      \text{for i with larger v, } & b_i \sim U(0, v_{second largest}) \\
                      \text{for j with smaller v, }& b_j = 
                      \begin{cases}
                         0  \text{ with 0.5 probability} \\
                         \sim U(0, v) \text{with 0.5 probability}
                      \end{cases}
                   \end{align*}
             \end{itemize}
          \item Expected revenue for seller = \[v_{second highest}\]
      \end{itemize}
   \item For All Pay Auction, imperfect info
      \begin{itemize}
         \item Equilibrium bidding strategy 
            \[
               b_i = (n-1)\frac{v_i^{n}}{n}
            \]
         \item Expected revenue for seller = \[\frac{n-1}{n+1}\]
      \end{itemize}
\end{enumerate}


\section{Bidding Strategy for First Price Sealed Bid}
\subsection{For 2 players}

It can be shown that for 2 bidders with $v_1, v_2, \sim U(0, 1)$, their equilibirum bidding strategy is uniquely $b_i = \frac{v_i}{2}$ \\

Assuming that the $b_2$ is bidding with function $b_2 = B_2(v_2)$ such that $b_2 = kv_2, k <= 1$

\begin{eqnarray}
u_1 & = & P(kv_2 \le b_1)(v_1-b_1) \nonumber \\
& = & P(v_2 \le \frac{b_1}{k})(v_1-b_1) \nonumber \\
& = & (\frac{b_1}{k})(v_1-b_1) \nonumber \text{ since $v_2 \sim U(0, 1), F_v(X) = X$}  
\end{eqnarray}
profits maximized when, $\frac{du_1}{db_1} = 0$ 

\begin{align*}
\frac{v_1 - b_1}{k} - \frac{b_1}{k}  = 0 \\
\frac{1}{k}(v_1 - 2b_1) = 0 \\
\frac{v_1}{2}  = {b_1} \\
\end{align*}

this means that for any of opponent's k, bidding $b_i = \frac{v_i}{2}$ is a NE

\subsection{Generalizing for $n$ players}
For $n$ players each with $v_i \sim U(0, 1)$, prove that the equilibrium bid is $b_i = \frac{n-1}{n}v_i$ \\

\subsubsection{Argument 1 - by profit max argument}
\begin{eqnarray}
u_1 & = & P(v_1 - b_1) \text{ where P is the probability of winning} \nonumber \\
& = & P(\text{bid of n-1 others $<\ b_1$ })(v_1 - b_1) \nonumber 
\text{ assuming everyone bidding with $b_j = kv_j$, } \\
& = & P(kv < b_1)^{n-1}(v_1 - b_1) \nonumber \\
& = &(\frac{b_1}{k})^{n-1}(v_1 - b_1)\nonumber
\end{eqnarray}

To maximize profits, $\frac{du_1}{db_1} = 0$ when
\begin{eqnarray}
   \frac{1}{k^{n-1}} (v_1(n-1)b^{n-2} - nb^{n-1}) & = & 0 \nonumber \\
   v_1(n-1)b^{n-2} - nb^{n-1} & = & 0 \nonumber \\
   b^{n-2} (v_1(n-1) - nb) & = & 0 \nonumber \\
   b & = & \frac{n-1}{n}v \nonumber
\end{eqnarray}

\subsubsection{Argument 2 - By Probability}
Given that in equilibrium, players bid the expected value of the second highest v, assuming that they themselves have the highest v, \\ \\
let $Y$ be the equilibrium bid $Y \sim max(v_2, v_3 ... v_n)\ given\ that\ v_2, v_3 ... v_n < v_1$, find $E(Y|Y \le v_1)$ \\

\begin{eqnarray}
   F_Y(y) & = & P(Y \le y) \nonumber \\
   & = & P(max(v_2, v_3 ... v_n) \le y) \nonumber \\
   & = & P(v_2, v_3 ... v_n \le y) \nonumber \\
   & = & (P(v_2 \le y))^{n-1} \nonumber \\
   & = & y^{n-1} \nonumber \text{ since $F_v(X) = X$ and all $v \sim U(0, 1)$}
\end{eqnarray}

To find the CDF of Y conditional on $Y < v_1$,
\begin{equation}
   F_Y(y |Y \le v_1) = \frac{P((Y \le y) \cap (Y \le v_1))}{P(Y \le v_1)} \nonumber \\
\end{equation}

Consider only case in which $ y \le v_1$, hence $(Y \le y) \subseteq (Y \le v_1)$

\begin{equation}
   \frac{P((Y \le y) \cap (Y \le v_1))}{P(Y \le v_1)} = \frac{P(Y \le y)}{P(Y \le v_1)}
   = \frac{y^{n-1}}{v_1^{n-1}} \nonumber
\end{equation}

The PDF of Y is thus

\begin{eqnarray}
   f_Y(y) & = & \frac{d}{dy}\frac{y^{n-1}}{v_1^{n-1}} \nonumber \\
   & = & (n-1)\frac{y^{n-2}}{v^{n-1}} \nonumber
\end{eqnarray}

The expected value of Y is 

\begin{align*}
   E(Y|Y \le v_1) = \int_{0}^{v_1} yf_Y(y|y \le v_1) dy \\
   = \frac{n-1}{v_1^{n-1}}\int_{0}^{v_1} y^{n-1} dy \\
   = \frac{n-1}{v^{n-1}} \left[\frac{y^n}{n}\right]^{v}_{0} \\
   = \frac{n-1}{v^{n-1}}\frac{v^n}{n} \\
   = \frac{n-1}{n} v
\end{align*}

\section{Expected Return for First Price Sealed Bid}
\subsection{Argument 1}
Knowing that in equilibrium, each bidder bids $b_i = B_i(v_i)$ where $B_i$ is the equilibrium bidding function, the expected revenue is $E(B_i(X))$ where $X$ is the highest v.
\subsubsection{Case with 2 players}

For 2 bidders with $v_1, \ v_2$, each bidding $b_i = \frac{v_i}{2}$, the expected revenue is the half of the expected highest v. \\

\begin{eqnarray}
   P(max(v_1, v_2) = x) & = & P(v_1 = x, v_2 < x) + P(v_2 = x, v_1 < x) \nonumber \\
   & = & 2P(v_1 = x, v_2 < x)  \nonumber \\
   & = & 2 f_v(x) F_v(x) \nonumber \\
   & = & 2x  \nonumber \text{ , since $v \sim U(0, 1)$ and $F_v(x) = x$}
\end{eqnarray}

The expected value of $max(v_1, v_2)$ is
\begin{eqnarray}
   E(f_{HV}(x)) = \int_0^1 x f_{HV}(x) dx & =  &\int_0^12x^2 dx \nonumber \\
   & = & 2 \left[\frac{x^3}{3}\right] ^1_0 \nonumber \\
   & = & \frac{2}{3} \nonumber 
\end{eqnarray}

Since expected revenue is half of expected highest v, 
\begin{equation}
   expected\ revenue = \frac{3}{2} \times \frac{1}{2} = \frac{1}{3} \nonumber
\end{equation}

\subsubsection{Generalization for n players}
For $n$ bidders with $v_i \in \left( v_1, v_2, v_3 ... v_n \right) $, given that in equilibrium, each player bids with $b_i = \frac{n-1}{n} v_i$ as shown earlier, the expected revenue is $\frac{n-1}{n}v_{max}$ where $v_{max} = max\left( v_1, v_2, v_3 ... v_n \right) $ is the expected highest v \\

let $max\left( v_1, v_2, v_3 ... v_n \right)$ have PDF $f_{hv}(x)$, where
\begin{eqnarray}
   f_{HV}(x) & = & P(max\left( v_1, v_2, v_3 ... v_n \right) = x) \nonumber \\
             & = & \sum_{i = 1}^{n} P(v_i = x,\  \left(v_1, v_2 ... v_n\right) / v_i \le x) \nonumber \\
             & = & nf_v(x)(F_v(x))^{n-1} \nonumber \\
             & = & nx^{n-1} \nonumber 
\end{eqnarray}

The expected value of $max\left( v_1, v_2, v_3 ... v_n \right)$ is thus 
\begin{align*}
   E(f_{HV}(x))  &=  \int_0^1 x f_{HV}(x) dx \\
                 &=  n\int_0^1x^n dx \\
                 &=x \left[\frac{x^{n+1}}{n+1} \right] ^1_0 \\
                 &= \frac{n}{n+1}
\end{align*}

Expected revenue is therefore 
\begin{align*}
   \frac{n-1}{n} \frac{n}{n+1} = \frac{n-1}{n+1}
\end{align*}

\subsection{Argument 2 - Prof Massi's argument}

\subsubsection{For 2 bidders}
Knowing that in equilibrium, bidders bid their expected value of the second highest v assuming that they themselves have the highest v, \\

Expected revenue = expected second highest v\\

For 2 bidders with $v_1\ v_2 \sim U(0,1)$, let $Y$ be $min(v_1, v_2)$. Find E(Y)\\  

The CDF of Y is 
\begin{align*}
   F_Y(y) &= P(min(v_1, v_2) \le y) \\
          &= P(v_1 \le y, v_2 \le y) + P(v_1 \le y, v_2 > y) + P(v_2 \le y, v_1 > y) \\
          &= y^2 + y(1-y) + y(1-y) \\
          &= 2y - y^2 \\
\end{align*} 
The CDF of Y can also be found by 
\begin{align*}
   F_Y(y) &= P(min(v_1, v_2) \le y) \\ 
          &= 1 - P(min(v_1, v_2) \ge y) \\
          &= 1 - P(v_1 \ge y, v_2 \ge y) \\
          &= 1 - (1-y)^2 \\
          &= 2y - y^2 \\
\end{align*}

The expected value of Y can thus be calculated from its PDF
\begin{align*}
   f_Y(y) &= \frac{d}{dy} (2y - y^2) \\ 
          &= 2 - 2y \nonumber 
\end{align*}

\begin{align*}
   E(Y) &= \int_0^1 y f_Y(y) dy = \int_0^1 2y - 2y^2 dy \\ 
        &= \frac{1}{3} \nonumber \\
\end{align*}

\subsubsection{Generalisation for n players}
For $n$ players with $v_i \in \left(v_1, v_2, v_3 ... v_n\right), v_i \sim U(0,1)$, the expected revenue is the expected second highest v.  \\

Hence assume $v_1 > v_i \in (v_2, v_3, ... v_n)$, let $Y$ be $max(v_2, v_3, ... v_n)$, find $E(Y|Y < v_1)$ 

\begin{align*}
   F_Y(y) &= P(Y \le y) \\
          &= P(max(v_2, v_3, ... v_n) \le y) \\ 
          &= y^{n-1} \\ 
   F_Y(y | Y < v_1) &= \frac{y^{n-1}}{v_1^{n-1}} \\
   f_Y(y | Y < v_1) &= (n-1)\frac{y^{n-2}}{v_1^{n-1}} 
\end{align*}

\begin{align*}
   E(Y | Y < v_1) &= \int_0^{v_1} \frac{n-1}{v_1^{n-1}} y^{n-2} y dy \nonumber \\
                  &= \frac{n-1}{v_1^{n-1}} \left[ \frac{y^n}{n}\right]^{v_1}_0 \\
                  &= \frac{n-1}{n} v_1
\end{align*}

Since $v_1$ is the maximum v, find let $X$ be $max(v_1, v_2, v_3 ... v_n)$, find $E(X)$

\begin{align*}
   F_X(x) &= P(max(v_1, v_x, v_3... v_n) \le x) \\
          &= x^n \\
   f_X(x) &= nx^{n-1} \\
   E(X) &= \int _0^1 n x^n dx  \\
        &= n \left[ \frac{x^{n+1}}{n+1}\right]^1_0 \\
        &= \frac{n}{n+1}
\end{align*}

Hence the expected revenue for seller is 
\[\frac{n-1}{n}\frac{n}{n+1} = \frac{n-1}{n+1}\]

Alternatively, the PDF of the second highest v can be found directly. Let the second highest v have PDF $f_{2nd\ highest}(x)$.

taken from http://econ.ucsb.edu/~tedb/Courses/GameTheory/aucnotes.pdf , although it is technically wrong to claim that 
\[ P(X = x) = f_X(x) \]

\begin{align*}
   f_{2nd\  highest}(x) &= P(second\ highest\ v = x) \\
                      &= P(\text{one person has $v = x$, one person has $v > x$, $n-2$ others have $v \le x$})
\end{align*}

probability of one person having $v = x$ is \[ P(v_i = x) = f_V(x) = 1\] \\
probability of one person having $v \ge x$ is 
\begin{align*}
  P(v_j \ge x)  &= 1 - P(x_j \le x)  \\
                &= 1 - F_V(x)
\end{align*}
probability of n-2 having $v \le x$ is \[ x^{n-2}\] \\
number of ways to choose 1 person to have $v = x$ and 1 to have $v > x$ is $n (n-1)$ \\

\begin{align*}
   f_{2nd\  highest}(x) &= P(\text{one person has $v = x$, one person has $v > x$, $n-2$ others have $v \le x$}) \\ 
     &= n(n-1)(1-F_V(x))x^{n-2} \\
     &= n(n-1)(1-x)x^{n-2} 
\end{align*}

the expected second highest v is thus

\begin{align*}
   E(f_{2nd\  highest}(x)) &= \int_0^1 x f_{2nd\  highest}(x) dx \\
                           &= n(n-1) \int_0^1 (1-x)x^{n-1} dx \\
                           &= n(n-1) \int_0^1 x^{n-1} - x{n} dx \\
                           &= n(n-1) \left[ \frac{x^n}{n} - \frac{x^{n+1}}{n+1}\right]_0^1 \\
                           &= n(n-1) \left( \frac{1}{n} - \frac{1}{n+1} \right) \\
                           &= n - 1 - \frac{n(n-1)}{n+1} \\
                           &= \frac{(n-1)(n+1) - n(n-1)}{n+1} \\
                           &= \frac{n-1}{n+1}
\end{align*}

\section{All Pay Auctions, imperfect information}
\subsection{Case for 2 players}

for 2 players with $v_1, v_2 \sim U(0, 1)$, find their equilibrium bid function $B$. Assume B is a function that is monotonic increasing, and is hence invertible and differentiable. 

Suppose bidder 1 is bidding $b_1 = B(x)$ 

\begin{align*}
   u_1 &= P(v_1 - b_1) + (1-P)(-b_1) \text{, where P is probability of winning with $b_1$} \\
       &= Pv_1 - Pb_1 - b_1 + Pb_1 \\
       &= Pv_1 - b_1  \\
       &= P(B(x) > B(v_2))(v_1) - B(x) \\
       &= = x(v_1) -B(x)
\end{align*}

B is the NE bid function if x = v satisfies the profit-max equation $\frac{du_1}{dx} = 0$, 
\begin{align*}
   v_1 - B'(x) &= 0 \\
   v_1 - B'(v_1) &= 0 \\
   v_1 &= B'(v_1) \\
   B(v_1) &= \int v_1 dv\\
          &= \frac{v^2}{2} + C
\end{align*}

with the condition that $B(0) = 0$, $C = 0$, hence the eqm bid function is
\[ 
   B(v_i) = \frac{v_i^2}{2}
\]


The expected revenue for seller is 

\begin{align*}
   E(R) &= E\left(\frac{v_1^2}{2} + \frac{v_2^2}{2}\right) \\
        &= \int_0^1 v^2 dv = \frac{1}{3}
\end{align*}

\section{Generalising for n players}
for n players with $v_i \in (v_1, v_2, v_3 ... v_n),\ v_i \sim U(0, 1)$, proof is similar to 2 player case above but expressed slightly differently. \\

As shown earlier the expected payoffs are 
\[
   u_i = Pv_i - b_i
\]

expressing payoffs in terms of $b_i$, a function $V$ can be defined as the inverse of equilibrium bidding function $B$, that is $V - B^{-1}$
\begin{align*} 
   b_i &= B(v_i)  \\
   v_i &= V(b_i) 
\end{align*}

hence to express $u_i$ in terms of $b_i$, 
\[
   u_i = P(b_i > max(b_1, b_2, b_3 ... b_n) /b_i )(v_i) - b_i 
\]

since everyone bidding with $B$ in equilibrium, and $B$ is monotonically increasing,
\begin{align*}
   P(b_i > max(b_1, b_2, b_3 .... b_n) / b_i ) &= \left(P(b_i > b_j)\right)^{n-1}  \\
                                               &= \left( P(V(b_i) > v)\right)^{n-1} \\
                                               &= \left(F_V(V(b_i))\right) ^{n-1}
                                               &= \left(V(b_i)\right)^{n-1}
\end{align*}

$u_i$ can thus be expressed as 
\[
   u_i = \left(V(b_i)\right)^{n-1}v_i - b_i 
\]

since the bidder chooses $b_i$ to maximize $u_i$, 
\begin{align*}
   \frac{du_i}{db_i} &= v_i(n-1)\left(V(b_i)\right)^{n-2}V^{'}(b_i) - 1 \\
                     &= (n-1)(v_i)^{n-1}V^{'}(b_i) - 1 \text{, since $v_i = V(b_i)$} \\
\end{align*}

since $V^{-1} = B$, $V^{'}(b_i) = \frac{1}{B^{'}(v_i)}$, hence 
\[
   \frac{du_i}{db_i} = \frac{(n-1)v_i^{n-1}}{B^{'}(v_i)} - 1
\]

profit maximized when 
\begin{align*}
   \frac{(n-1)v_i^{n-1}}{B^{'}(v_i)} &= 1 \\
   (n-1)v_i^{n-1} &= B^{'}(v_i) \\
   B(v_i) &= \int (n-1)v_i^{n-1} dv \\
          &= \frac{(n-1)v^n_i}{n}  + C. 
\end{align*}

Given additional condition that $B(0) = 0$, hence $C = 0$ and 
\[
  b_i = B(v_i) = \frac{(n-1)v^n_i}{n} 
\]

the expected revenue is thus 
\begin{align*}
   E\left(\sum_{i = 1}^{i = n} \frac{(n-1)v_i^n}{n} \right) &= \int_0^1 (n-1)v^ndv \\
                                                            &= (n-1)\left[\frac{v^{n+1}}{n+1}\right]_0^1 \\
                                                            &= \frac{n-1}{n+1}
\end{align*}

\section{Generalizing for n players, with different uniform distribution}
for n players each with $v_i \sim U(0, a)$, to show that normalizing works

\begin{align*}
   u_1 &= Pv_1 - b_1 \\
       &= \left(P(b_1 > b_i)\right)^{n-1}(v_1) - b_1
\end{align*}	

Given equilibrium bidding function and its inverse
\[
   b_i = B(v_i)
\]
\[
   v_i = V(b_i)
\]

\begin{align*}
   u_1 &= \left(P(B(v_i) < b_1)\right)^{n-1}v_1 - b_1 \\
       &= \left(P(v_i < V(b_1))\right)^{n-1}v_1 - b_1 \\
\end{align*}	

since the CDF of $v$ is now
\[
   F_V(x) = P(v \le x) = \frac{x}{a}
\]

\begin{align*}
   u_1 &= \left[ \frac{V(b_1)}{a}\right]^{n-1}v_1 - b_1 \\
   \frac{du_1}{db_1} &= (n-1) \left[ \frac{V(b_1)}{a}\right]^{n-2} \frac{V'(b_1)}{a}v_1 - 1 \\
                     &= (n-1) \left(\frac{1}{a}\right)^{n-1}v_1^{n-1} \frac{1}{B'(v_1)} - 1
\end{align*}	

$u_1$ max when 
\[
   B'(v_1) = (n-1)\left(\frac{1}{a}\right)^{n-1}v_1^{n-1} 
\]
\[
   B(v_1) = \int(n-1)\left(\frac{1}{a}\right)^{n-1}v_1^{n-1} dv
\]
\[
   B(v_1) = (n-1)\left(\frac{1}{a}\right)^{n-1}\frac{v_1^n}{n}
\]

hence the expected revenue is
\begin{align*}
   E(R) &= \int_0^a B(v_1) \times n \times f(v) dv \\
\end{align*}	
since f(v) is the pdf of v 
\[
   f(v) = \frac{1}{a}
\]
\begin{align*}
   E(R) &= \int_0^a B(v_1) \times n \times f(v) dv \\
        &= \int_0^a(n-1)\left[\frac{1}{a}\right]^{n}v_1^n dv \\
        &= (n-1)\left(\frac{1}{a}\right)^{n} \left[\frac{v^{n+1}}{n+1}\right]_0^a \\
        &= \frac{n-1}{n+1} a 
\end{align*}	

\end{document}
