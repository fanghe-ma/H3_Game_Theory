\documentclass[a4paper, 10pt]{article}
\usepackage[margin = 1in]{geometry}
\usepackage{amsmath}
\usepackage{tabularx}
\usepackage{framed}
\setlength{\parindent}{0em}

\begin{document}

\section*{Repeated games}

\section{twice repeated PD}
strategy profile resemble

\[
   { X, \left(X, X, X, X\right) }
\]

where X is strategy in first round, and $(X, X, X, X)$ are strategies in second round for each of the 4 histories that could arise in first round, there are thus $2^5 = 32$ different strategies \\

let payoffs be 
\[
G > C > D > B
\]


The payoffs are 
\[
   U = \sum_{t = 0}^{T-1} \delta^t g_{t+1}
\]

\section{One Shot deviation principle}
\begin{framed}
   a strategy profile is SGPE if and only if at \textbf{any subgame}, no player can improve her payoff by change her action from said profile for one round only, ie no profitable one shot deviation
\end{framed}	
\begin{itemize}
   \item finding a SGPE thus entails finding the consequences of a one shot deviation at stage $t$ knowing that at stage $t+1$ players will play the specified strategy profile
   \item deviation must not be profitable for all $t$ 
\end{itemize}	

\section{Grim trigger}
\begin{framed}
   the grim trigger strategy: 
   \begin{itemize}
      \item cooperate and remain in cooperative state until there is a deviation
      \item remain in punishment stage and defect forever
   \end{itemize}	
\end{framed}	

\begin{center}
   \begin{tabular}{c |c c c c c c c c c c}
      standard & C & C & C & C & C & C & C & C & C & C \\
               & C & C & C & C & C & C & C & C & C & C \\
     \hline
      with defection & C & C & C & C & D & D & D & D & D & D \\
                     & C & C & C & C & C & D & D & D & D & D \\
   \end{tabular}
\end{center}  


each stage thus falls into one of two categories
\begin{itemize}
   \item cooperative state: no defect yet
      \[
         V(Coop) = \sum_{i=0}^{\infty} C \delta^i = \frac{C}{1 - \delta}
      \]
   \item punishment state: defect at least once before
      \[
         V(Pun) = \sum_{i=0}^{\infty} D \delta^i = \frac{D}{1- \delta}
      \]
      
\end{itemize}	

To test for SGPE
\begin{enumerate}
   \item first test for profitability of deviation under cooperative state
         \[
         V(Coop) = V(Coop)
         \]
         \[
            V(Defect) = G + \delta V(Pun)
         \]
         
   \item then test for profitability of deviation under punishment state
      \[
         V(Coop) = B + \delta  V(Pun)
      \]
      \[
         V(Defect) = V(Pun)
      \]
      
\end{enumerate}	

Alternatively, the consequences of deviating under cooperative state could be expressed as 
\[
   \sum \text{ gain from first round + losses for all future rounds}
\]
\[
  \sum (G-C) + \delta (D-C) + \delta^2 (D-C) + ... \delta^{\infty} (D-C) \ge 0
\]
\[
 (G-C) + \frac{\delta (D-C)}{1 - \delta } \ge 0
\]
\[
   (G-C) \ge \frac{\delta (C-D)}{1 - \delta}
\]


\section{Tit for tat}
\begin{framed}
   The TFT strategy:
   \begin{itemize}
      \item start in mutual cooperation, then each player copies the move of the opponent in the previous round
   \end{itemize}	
\end{framed}	

\begin{center}
   \begin{tabular}{c |c c c c c c c c c c}
      standard & C & C & C & C & C & C & C & C & C & C \\
               & C & C & C & C & C & C & C & C & C & C \\
     \hline
      with defection & C & C & C & C & D & C & D & C & D & C \\
                     & C & C & C & C & C & D & C & D & C & D \\
   \end{tabular}
\end{center}  

each stage thus falls into one of four categories, let $V(X, X)$ denote the state of a stage where X, X were the strategies for the previous round
\begin{itemize}
   \item cooperative state: last round was C, C
      \[
         V(C, C) = \sum_{i=0}^{\infty} C \delta^i = \frac{C}{1 - \delta}
      \]
   \item punishment state: last round was D, D 
      \[
         V(D, D) = \sum_{i=0}^{\infty} D \delta^i = \frac{D}{1- \delta}
      \]
   \item alternating state 1: last round was C, D
      \[
      V(C, D) = \frac{G + B \delta}{1-\delta^2}
      \]
   \item alternating state 1: last round was D, C
      \[
      V(D, C) = \frac{B + G \delta}{1-\delta^2}
      \]
\end{itemize}	

now test for profitability of deviation under each stage
\begin{itemize}
   \item under state V(C, C)
      \begin{itemize}
         \item TFT strategy is to play C, for a payoff of 
            \[
               V(C, C)
            \]
            
         \item deviation strategy is to play D, stage effectively becomes $V(C, D)$  (as if opponent played D in the previous round, payoffs are
            \[
               V(C, D)
            \]
         hence first condition for TFT sustaining cooperation as SGPE is 
         \begin{equation}
            V(C, C) \ge V(C, D)
         \end{equation}	
            
      \end{itemize}	

   \item under state V(D, D)
      \begin{itemize}
         \item TFT strategy is to play D, for a payoff of 
            \[
               V(D, D)
            \]
            
         \item deviation strategy is to play C, stage effectively becomes $V(D, C)$  (as if opponent played D in the previous round, payoffs are
            \[
               V(D, C)
            \]
         hence second condition for TFT sustaining cooperation as SGPE is 
         \begin{equation}
            V(D, D) \ge V(D, C)
         \end{equation}	
            
      \end{itemize}	

   \item under state V(C, D)
      \begin{itemize}
         \item TFT strategy is to play D, for a payoff of 
            \[
               V(C, D)
            \]
            
         \item deviation strategy is to play C, stage effectively becomes $V(C, C)$  (as if opponent played D in the previous round, payoffs are
            \[
               V(C, C)
            \]
         hence second condition for TFT sustaining cooperation as SGPE is 
         \begin{equation}
            V(C, D) \ge V(C, C)
         \end{equation}	
            
      \end{itemize}	
   \item under state V(D, C)
      \begin{itemize}
         \item TFT strategy is to play C, for a payoff of 
            \[
               V(D, C)
            \]
            
         \item deviation strategy is to play D, stage effectively becomes $V(D, D)$  (as if opponent played D in the previous round, payoffs are
            \[
               V(D, D)
            \]
         hence second condition for TFT sustaining cooperation as SGPE is 
         \begin{equation}
            V(D, C) \ge V(D, D)
         \end{equation}	
            
      \end{itemize}	
\end{itemize}	

hence TFT is a SGPE if and only if 
\begin{align*}
   V(C, C) = V(C, D) \\
   V(D, D) = V(D, C)
\end{align*}	
\section{Stick and carrorts}
\begin{framed}
   Stick and carrots strategy: 
   \begin{itemize}
      \item define punishment state as playing minimax, which is the minimum payoff that one player can induce to the other
      \item more specifically, this entails P1 playing best response to P2, and P2 minimizing $\Pi_1$ wrt $p2$ 
      \item play C if there is no defection
      \item punish for t rounds and then return to cooperation, unless there is deviation in which case restart the clock
   \end{itemize}	
\end{framed}	

given the collusive pricing example, 
\begin{itemize}
   \item profit function is
      \[
         \Pi_i = (p_i - 8)(44 - 2p_i +p_2)
      \]
   \item BR for price is
      \[
         p_i = \frac{60 + p_2}{4}
      \]
   \item under collusion / cooperation, 
      \[
      p_1 = p_2 = 26, \ \Pi_1 = \Pi_2 = 324
      \]
   \item under NE 
      \[
      p_1 = p_2 = 20, \ \Pi_1 = \Pi_2 = 288
      \]
   \item under minimax, 
      \[
      p_1 = p_2 = 8, \ \Pi_1 = \Pi_2 = 0
      \]
\end{itemize}	
hence game resembles
\begin{center}
   \begin{tabular}{c |c c c c c c c c c c}
      standard & 28 & 28& 28& 28& 28& 28& 28& 28& 28 \\
               & 28 & 28& 28& 28& 28& 28& 28& 28& 28 \\
     \hline
      with defection & 28 & 28& 28& 28& 28& 21.5& 8& 28& 28 \\
                     & 28 & 28& 28& 28& 28& 28& 8& 28& 28 \\
     \hline
      with defection in punsihment state 
                     & 28 & 28& 28& 28& 28& 21.5& 8& 8& 28 \\
                     & 28 & 28& 28& 28& 28& 28& 17& 8& 28 \\
   \end{tabular}
\end{center}  
      
the value of cooperative state
\[
   V(C) = 324 + \delta V(C) = \frac{324}{1- \delta}
\]
the value of punsihment state
\[
   V(D) = 0 + \delta V(C) = \delta \frac{324}{1- \delta}
\]
to test for SGPE
\begin{itemize}
   \item under cooperative state
      \begin{itemize}
         \item SC strategy is to play 26
            \[
               V(SC) = V(C) = 324 + \delta V(C)
            \]
            
         \item deviation would be to play BR to 26, which is $21.5, $ giving payoff $\frac{729}{2}$ in the current round
            \[
               V(dev) = \frac{729}{2} + \delta V(D) = \frac{729}{2} + \delta^2 V(C)
            \]
            hence deviation not profitable if 
           \begin{equation}
              324 + \delta V(C) \ge \frac{729}{2} + \delta^2 V(C)
           \end{equation}	 
           \[
           \delta V(C) (1 - \delta) \ge \frac{729}{2} - 324 
           \]
           \[
              \delta \ge \frac{40.5}{324}
           \]
      \end{itemize}	
   \item under punishment state
      \begin{itemize}
         \item SC strategy is to play 8
            \[
               V(SC) = V(D) = \delta V(C)
            \]
          \item deviation strategy is to deviate from 8 and play BR to opponent's 8, playing 17, giving a payoff of 162
             \[
                V(dev) = 162 + \delta V(D)
             \]
            hence deviation not profitable if
           \begin{equation}
              V(D) \ge 162 + \delta V(D)
           \end{equation}	 
           \[
              V(D) (1- \delta) \ge 162
           \]
           \[
           \delta V(C) (1 - \delta) \ge 162
           \]
           \[
              \delta 324 \ge 162
           \]
      \end{itemize}	
\end{itemize}	

\end{document}	
